%% start of file `moderncvstylebanking.sty'.
%% Copyright 2006-2013 Xavier Danaux (xdanaux@gmail.com).
%
% This work may be distributed and/or modified under the
% conditions of the LaTeX Project Public License version 1.3c,
% available at http://www.latex-project.org/lppl/.


%-------------------------------------------------------------------------------
%                identification
%-------------------------------------------------------------------------------
\NeedsTeXFormat{LaTeX2e}
\ProvidesPackage{moderncvstylebanking}[2013/04/29 v1.5.1 modern curriculum vitae and letter style scheme: banking]


%-------------------------------------------------------------------------------
%                required packages
%-------------------------------------------------------------------------------


%-------------------------------------------------------------------------------
%                overall style definition
%-------------------------------------------------------------------------------
% fonts
%\ifxetexorluatex
%  \setmainfont{Tex-Gyre Pagella}
%  \setsansfont{Tex-Gyre Pagella}
%  \setmathfont{Tex-Gyre Pagella}
%  \setmathfont[range=\mathit,\mathsfit]{Tex-Gyre Pagella Italic}
%  \setmathfont[range=\mathbfup,\mathbfsfup]{Tex-Gyre Pagella Bold}
%  \setmathfont[range=\mathbfit,\mathbfsfit]{Tex-Gyre Pagella Bold Italic}
%\else
  \IfFileExists{tgpagella.sty}%
    {%
      \RequirePackage{tgpagella}%
      \renewcommand*{\familydefault}{\rmdefault}}%
    {}
%\fi

% symbols
\moderncvicons{awesome}

% commands
\newcommand*{\maketitlesymbol}{%
    {~~~{\rmfamily\textbullet}~~~}}% the \rmfamily is required to force Latin Modern fonts when using sans serif, as OMS/lmss/m/n is not defined and gets substituted by OMS/cmsy/m/n
%   internal command to add an element to the footer
%   it collects the elements in a temporary box, and checks when to flush the box
\newsavebox{\maketitlebox}%
\newsavebox{\maketitletempbox}%
\newlength{\maketitlewidth}%
\newlength{\maketitleboxwidth}%
\newif\if@firstmaketitleelement\@firstmaketitleelementtrue%
%   adds an element to the maketitle, separated by maketitlesymbol
%   usage: \addtomaketitle[maketitlesymbol]{element}
\newcommand*{\addtomaketitle}[2][\maketitlesymbol]{%
  \if@firstmaketitleelement%
    \savebox{\maketitletempbox}{\usebox{\maketitlebox}#2}%
  \else%
    \savebox{\maketitletempbox}{\usebox{\maketitlebox}#1#2}\fi%
  \settowidth{\maketitleboxwidth}{\usebox{\maketitletempbox}}%
  \ifnum\maketitleboxwidth<\maketitlewidth%
    \savebox{\maketitlebox}{\usebox{\maketitletempbox}}%
    \@firstmaketitleelementfalse%
  \else%
    \flushmaketitle{}\\%
    \savebox{\maketitlebox}{#2}%
    \savebox{\maketitletempbox}{#2}%
    \settowidth{\maketitleboxwidth}{\usebox{\maketitlebox}}%
    \@firstmaketitleelementfalse\fi}
%   internal command to flush the maketitle
\newcommand*{\flushmaketitle}{%
  \strut\usebox{\maketitlebox}%
  \savebox{\maketitlebox}{}%
  \savebox{\maketitletempbox}{}%
  \setlength{\maketitleboxwidth}{0pt}}
\renewcommand*{\maketitle}{%
  \setlength{\maketitlewidth}{0.8\textwidth}%
  \hfil%
  \parbox{\maketitlewidth}{%
    \centering%
    % name and title
    \namestyle{\@firstname~\@lastname}%
    \ifthenelse{\equal{\@title}{}}{}{\titlestyle{~|~\@title}}\\% \isundefined doesn't work on \@title, as LaTeX itself defines \@title (before it possibly gets redefined by \title) 
    % detailed information
    \addressfont\color{color2}%
    \ifthenelse{\isundefined{\@addressstreet}}{}{\addtomaketitle{\addresssymbol\@addressstreet}%
      \ifthenelse{\equal{\@addresscity}{}}{}{\addtomaketitle[~--~]{\@addresscity}}% if \addresstreet is defined, \addresscity and \addresscountry will always be defined but could be empty
      \ifthenelse{\equal{\@addresscountry}{}}{}{\addtomaketitle[~--~]{\@addresscountry}}%
      \flushmaketitle\@firstmaketitleelementtrue\\}%
    \collectionloop{phones}{% the key holds the phone type (=symbol command prefix), the item holds the number
      \addtomaketitle{\csname\collectionloopkey phonesymbol\endcsname\collectionloopitem}}%
    \ifthenelse{\isundefined{\@email}}{}{\addtomaketitle{\emailsymbol\emaillink{\@email}}}%
    \ifthenelse{\isundefined{\@homepage}}{}{\addtomaketitle{\homepagesymbol\httplink{\@homepage}}}%
    \collectionloop{socials}{% the key holds the social type (=symbol command prefix), the item holds the link
      \addtomaketitle{\csname\collectionloopkey socialsymbol\endcsname\collectionloopitem}}%
    \ifthenelse{\isundefined{\@extrainfo}}{}{\addtomaketitle{\@extrainfo}}%
    \flushmaketitle}\\[2.5em]}% need to force a \par after this to avoid weird spacing bug at the first section if no blank line is left after \maketitle


%-------------------------------------------------------------------------------
%                resume style definition
%-------------------------------------------------------------------------------
% fonts
\renewcommand*{\namefont}{\Huge\bfseries\upshape}
\renewcommand*{\titlefont}{\Huge\mdseries\upshape}
\renewcommand*{\addressfont}{\normalsize\mdseries\upshape}
\renewcommand*{\quotefont}{\large\slshape}
\renewcommand*{\sectionfont}{\Large\bfseries\upshape}
\renewcommand*{\subsectionfont}{\large\upshape\fontseries{sb}\selectfont}
\renewcommand*{\hintfont}{\bfseries}

% styles
\renewcommand*{\namestyle}[1]{{\namefont\textcolor{color1}{#1}}}
\renewcommand*{\titlestyle}[1]{{\titlefont\textcolor{color2!85}{#1}}}
\renewcommand*{\addressstyle}[1]{{\addressfont\textcolor{color1}{#1}}}
\renewcommand*{\quotestyle}[1]{{\quotefont\textcolor{color1}{#1}}}
\renewcommand*{\sectionstyle}[1]{{\sectionfont\textcolor{color1}{#1}}}
\renewcommand*{\subsectionstyle}[1]{{\subsectionfont\textcolor{color1}{#1}}}
\renewcommand*{\hintstyle}[1]{{\hintfont\textcolor{color0}{#1}}}

% lengths
\newlength{\quotewidth}
\newlength{\hintscolumnwidth}
\setlength{\hintscolumnwidth}{0.3\textwidth}%
\newlength{\separatorcolumnwidth}
\setlength{\separatorcolumnwidth}{0.025\textwidth}%
\newlength{\maincolumnwidth}
\newlength{\doubleitemcolumnwidth}
\newlength{\listitemsymbolwidth}
\settowidth{\listitemsymbolwidth}{\listitemsymbol}
\newlength{\listitemmaincolumnwidth}
\newlength{\listdoubleitemmaincolumnwidth}

% commands
\renewcommand*{\recomputecvlengths}{%
  \setlength{\quotewidth}{0.65\textwidth}%
  % main lenghts
  \setlength{\maincolumnwidth}{\textwidth}%
  % listitem lengths
  \setlength{\listitemmaincolumnwidth}{\maincolumnwidth-\listitemsymbolwidth}%
  % doubleitem lengths
  \setlength{\doubleitemcolumnwidth}{\maincolumnwidth-\separatorcolumnwidth}%
  \setlength{\doubleitemcolumnwidth}{0.5\doubleitemcolumnwidth}%
  % listdoubleitem lengths
  \setlength{\listdoubleitemmaincolumnwidth}{\maincolumnwidth-\listitemsymbolwidth-\separatorcolumnwidth-\listitemsymbolwidth}%
  \setlength{\listdoubleitemmaincolumnwidth}{0.5\listdoubleitemmaincolumnwidth}%
  % fancyhdr lengths
  \renewcommand{\headwidth}{\textwidth}%
  % regular lengths
  \setlength{\parskip}{0\p@}}

\renewcommand*{\makecvtitle}{%
  % recompute lengths (in case we are switching from letter to resume, or vice versa)
  \recomputecvlengths%
  \maketitle%
  % optional quote
  \ifthenelse{\isundefined{\@quote}}%
    {}%
    {{\centering\begin{minipage}{\quotewidth}\centering\quotestyle{\@quote}\end{minipage}\\[2.5em]}}%
  \par}% to avoid weird spacing bug at the first section if no blank line is left after \maketitle}

\RenewDocumentCommand{\section}{sm}{%
  \par\addvspace{2.5ex}%
  \phantomsection{}% reset the anchor for hyperrefs
  \addcontentsline{toc}{section}{#2}%
  \strut\sectionstyle{#2}%
  {\color{color1}\hrule}%
  \par\nobreak\addvspace{1ex}\@afterheading}

\newcommand{\subsectionfill}{\xleaders\hbox to 0.35em{\scriptsize.}\hfill}% different subsectionfills will not be perfectly aligned, but remaining space at the end of the fill will be distributed evenly between leaders, so it will be barely visible
\RenewDocumentCommand{\subsection}{sm}{%
  \par\addvspace{1ex}%
  \phantomsection{}%
  \addcontentsline{toc}{subsection}{#2}%
  \strut\subsectionstyle{#2}{\color{color1}{\subsectionfill}}%
  \par\nobreak\addvspace{0.5ex}\@afterheading}

\renewcommand*{\cvitem}[3][.25em]{%
  \ifthenelse{\equal{#2}{}}{}{\hintstyle{#2}: }{#3}%
  \par\addvspace{#1}}

\renewcommand*{\cvdoubleitem}[5][.25em]{%
  \begin{minipage}[t]{\doubleitemcolumnwidth}\hintstyle{#2}: #3\end{minipage}%
  \hfill% fill of \separatorcolumnwidth
  \begin{minipage}[t]{\doubleitemcolumnwidth}\ifthenelse{\equal{#4}{}}{}{\hintstyle{#4}: }#5\end{minipage}%
  \par\addvspace{#1}}

\renewcommand*{\cvlistitem}[2][.25em]{%
  \listitemsymbol\begin{minipage}[t]{\listitemmaincolumnwidth}#2\end{minipage}%
  \par\addvspace{#1}}

\renewcommand*{\cvlistdoubleitem}[3][.25em]{%
  \cvitem[#1]{}{\listitemsymbol\begin{minipage}[t]{\listdoubleitemmaincolumnwidth}#2\end{minipage}%
  \hfill% fill of \separatorcolumnwidth
  \ifthenelse{\equal{#3}{}}%
    {}%
    {\listitemsymbol\begin{minipage}[t]{\listdoubleitemmaincolumnwidth}#3\end{minipage}}}}

\renewcommand*{\cventry}[7][.25em]{
  \begin{tabular*}{\textwidth}{l@{\extracolsep{\fill}}r}%
	  {\bfseries #4} & {\bfseries #5} \\%
	  {\itshape #3\ifthenelse{\equal{#6}{}}{}{, #6}} & {\itshape #2}\\%
  \end{tabular*}%
  \ifx&#7&%
    \else{\\\vbox{\small#7}}\fi%
  \par\addvspace{#1}}

\newbox{\cvitemwithcommentmainbox}
\newlength{\cvitemwithcommentmainlength}
\newlength{\cvitemwithcommentcommentlength}
\renewcommand*{\cvitemwithcomment}[4][.25em]{%
  \savebox{\cvitemwithcommentmainbox}{\ifthenelse{\equal{#2}{}}{}{\hintstyle{#2}: }#3}%
  \setlength{\cvitemwithcommentmainlength}{\widthof{\usebox{\cvitemwithcommentmainbox}}}%
  \setlength{\cvitemwithcommentcommentlength}{\maincolumnwidth-\separatorcolumnwidth-\cvitemwithcommentmainlength}%
  \begin{minipage}[t]{\cvitemwithcommentmainlength}\ifthenelse{\equal{#2}{}}{}{\hintstyle{#2}: }#3\end{minipage}%
  \hfill% fill of \separatorcolumnwidth
  \begin{minipage}[t]{\cvitemwithcommentcommentlength}\raggedleft\small\itshape#4\end{minipage}%
  \par\addvspace{#1}}

\renewenvironment{thebibliography}[1]%
  {%
    \bibliographyhead{\refname}%
%    \small%
    \begin{list}{\bibliographyitemlabel}%
      {%
        \setlength{\topsep}{0pt}%
        \setlength{\labelwidth}{0pt}%
        \setlength{\labelsep}{0pt}%
        \leftmargin\labelwidth%
        \advance\leftmargin\labelsep%
        \@openbib@code%
        \usecounter{enumiv}%
        \let\p@enumiv\@empty%
        \renewcommand\theenumiv{\@arabic\c@enumiv}}%
        \sloppy\clubpenalty4000\widowpenalty4000%
%        \sfcode`\.\@m%
%        \sfcode `\=1000\relax%
  }%
  {%
    \def\@noitemerr{\@latex@warning{Empty `thebibliography' environment}}%
    \end{list}%
  }


%-------------------------------------------------------------------------------
%                letter style definition
%-------------------------------------------------------------------------------
% commands
\renewcommand*{\recomputeletterlengths}{
  \recomputecvlengths%
  \setlength{\parskip}{6\p@}}

\renewcommand*{\makelettertitle}{%
  % recompute lengths (in case we are switching from letter to resume, or vice versa)
  \recomputeletterlengths%
  % sender block
  \maketitle%
  \par%
   % recipient block
  \begin{minipage}[t]{.5\textwidth}
    \raggedright%
    \addressfont%
    {\bfseries\upshape\@recipientname}\\%
    \@recipientaddress%
  \end{minipage}
  % date
  \hfill % US style
%  \\[1em] % UK style
  \@date\\[2em]% US informal style: "April 6, 2006"; UK formal style: "05/04/2006"
  % opening
  \raggedright%
  \@opening\\[1.5em]%
  % ensure no extra spacing after \makelettertitle due to a possible blank line
%  \ignorespacesafterend% not working
  \hspace{0pt}\par\vspace{-\baselineskip}\vspace{-\parskip}}

\renewcommand*{\makeletterclosing}{
  \@closing\\[3em]%
  {\bfseries \@firstname~\@lastname}%
  \ifthenelse{\isundefined{\@enclosure}}{}{%
    \\%
    \vfill%
    {\color{color2}\itshape\enclname: \@enclosure}}}


\endinput


%% end of file `moderncvstylebanking.sty'.
